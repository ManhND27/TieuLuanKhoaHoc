\par Xử lý tín hiệu là một đề tài đóng vai trò rất quan trọng trong các ngành kĩ thuật. Tín hiệu mà chúng ta thu được có thể là âm thanh, hình ảnh,... Âm thanh là kết quả của sự dao động trong màng nhĩ lỗ tai, được mô phỏng bằng sóng âm lan truyền trong không khí phát ra từ tai nghe, âm thanh nhạc cụ, tiếng nói của con người,... Khi vẽ đồ thị những dao động này theo cường độ hay áp suất theo thời gian, ta được biểu diễn hình ảnh của âm thanh. Bên cạnh đó, tín hiệu biểu diễn cho hình ảnh là một hàm biến đổi, tuy nhiên, hàm biến đổi này không biến đổi theo thời gian mà biến đổi theo không gian hai chiều của ảnh. Về cơ bản có thể chia thành hai loại tín hiệu đó là tín hiệu liên tục và tín hiệu rời rạc. 
\par Để xử lý tín hiệu, chúng ta có nhiều phương pháp biến đổi tín hiệu nhằm biểu diễn dưới các miền không gian khác nhau như biến đổi Cosine, biến đổi Wavelet,... và không thể không kể đến biến đổi Fourier. Phương pháp biến đổi Fourier được đặt dựa theo tên của nhà toán học người Pháp Joseph Fourier, thực hiện việc biến đổi tín hiệu từ miền thời gian (chủ yếu là tín hiệu liên tục), hoặc miền không gian (chủ yếu là tín hiệu rời rạc) sang miền tần số. Biến đổi Fourier được sử dụng khi muốn sử dụng các đặc điểm hình học của một tín hiệu trong miền không gian vì tín hiệu trong miền tần số được phân rã thành các thành phần hình sin của nó, nên rất dễ kiểm tra hoặc xử lý các tần số nhất định, từ đó gây ra ảnh hưởng đến cấu trúc hình học trong miền không gian.
\par Trên cơ sở đó, nội dung đồ án của em gồm ba chương:
\begin{itemize}
    \item Chương 1 trình bày về kiến thức cơ sở. Tìm hiểu về phép biến đổi Fourier (nguồn gốc, xuất xứ,...), các tính chất và đặc điểm nói chung về mặt toán học và nói riêng về lĩnh vực xử lý ảnh số. Giới thiệu tổng quan về sự phát triển của các ứng dụng dùng phép biến đổi Fourier trong xử lý ảnh theo thời gian.
    \item Chương 2 trình này một số ứng dụng của biến đổi Fourier và lập trình bằng ngôn ngữ MATLAB trên các kiểu ảnh số khác nhau như ảnh màu, ảnh cấp xám, ảnh y tế,...
    \item Chương 3 trình bày một ứng dụng biến đổi Fourier trong nén ảnh số được đưa ra trong bài báo [25] của các tác giả Mohammed H. Rasheed và đồng nghiệp công bố năm 2020.
\end{itemize}
